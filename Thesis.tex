% LTeX: language=en-US
\documentclass{mimosis}

% The default language is English.
% If your thesis is in German, uncomment this line:
% \AtBeginDocument{\selectlanguage{ngerman}}

\usepackage{metalogo} % unnecessary
\usepackage{lipsum}   % unnecessary

% This can be used to typeset bold font in table while still aligning the numbers correctly:
\usepackage{etoolbox}

% Hyperlinks
\definecolor{GoetheBlue}{rgb}{0,0.38,0.56}
\usepackage[%
  colorlinks = true,
  citecolor  = GoetheBlue,
  linkcolor  = GoetheBlue,
  urlcolor   = GoetheBlue,
  unicode,
  ]{hyperref}
\renewcommand\UrlFont{\normalfont\small}
\renewcommand\path[1]{{\normalfont\small\detokenize{#1}}}
\usepackage[nameinlink,capitalise]{cleveref}

\usepackage{bookmark}

%%%%%%%%%%%%%%%%%%%%%%%%%%%%%%%%%%%%%%%%%%%%%%%%%%%%%%%%%%%%%%%%%%%%%%%%
% Fonts
%%%%%%%%%%%%%%%%%%%%%%%%%%%%%%%%%%%%%%%%%%%%%%%%%%%%%%%%%%%%%%%%%%%%%%%%

% Set serif font:
\usepackage[lining,semibold,scaled=1.05]{ebgaramond}

% Set monospace font:
\usepackage[scale=0.85]{sourcecodepro}

% if amsthm is needed, must be loaded before newtxmath
\usepackage{amsthm}

% Set math font:
\usepackage[ebgaramond,vvarbb,subscriptcorrection]{newtxmath}

% load after all math to give access to bold math using \bm{..} command:
\usepackage{bm}

% Acronym and glossary directories are rare in theoretical computer science.
% If possible, try to avoid them.
% \newacronym[description={Principal component analysis}]{PCA}{PCA}{principal component analysis}
% \newacronym                                            {SNF}{SNF}{Smith normal form}
% \newacronym[description={Topological data analysis}]   {TDA}{TDA}{topological data analysis}

% \newglossaryentry{LaTeX}{%
%   name        = {\LaTeX},
%   description = {A document preparation system},
%   sort        = {LaTeX},
% }

% \newglossaryentry{Real numbers}{%
%   name        = {$\real$},
%   description = {The set of real numbers},
%   sort        = {Real numbers},
% }

% \makeindex
% \makeglossaries

%%%%%%%%%%%%%%%%%%%%%%%%%%%%%%%%%%%%%%%%%%%%%%%%%%%%%%%%%%%%%%%%%%%%%%%%
% Ordinals
%%%%%%%%%%%%%%%%%%%%%%%%%%%%%%%%%%%%%%%%%%%%%%%%%%%%%%%%%%%%%%%%%%%%%%%%

\makeatletter
\@ifundefined{st}{%
  \newcommand{\st}{\textsuperscript{\textup{st}}\xspace}
}{}
\@ifundefined{rd}{%
  \newcommand{\rd}{\textsuperscript{\textup{rd}}\xspace}
}{}
\@ifundefined{nd}{%
  \newcommand{\nd}{\textsuperscript{\textup{nd}}\xspace}
}{}
\makeatother

\renewcommand{\th}{\textsuperscript{\textup{th}}\xspace}

%%%%%%%%%%%%%%%%%%%%%%%%%%%%%%%%%%%%%%%%%%%%%%%%%%%%%%%%%%%%%%%%%%%%%%%%
% Incipit
%%%%%%%%%%%%%%%%%%%%%%%%%%%%%%%%%%%%%%%%%%%%%%%%%%%%%%%%%%%%%%%%%%%%%%%%

\title{Very important thesis title}
\subtitle{Less important thesis subtitle}
\author{John Doe}
\date{\today}

\begin{document}

  \frontmatter
  \include{Sources/Title_Goethe_Uni}
  \include{Sources/Abstract}
  \tableofcontents

  \mainmatter
  \include{Sources/Introduction}
  \chapter{Preliminaries}

Try to make your thesis self-contained!
You should include here all the preliminary material (notation, definitions, etc.) that is necessary to understand your thesis.
As a rule of thumb, you can assume that your reader is a competent computer scientist who has finished their Master's degree.
For example, the reader knows all standard undergraduate material in theoretical computer science and/or discrete mathematics.
However, you cannot assume that the reader knows the notation you are going to use or has read any specific paper that you may have read.

For example, you can assume that the reader knows what a graph is and how Dijkstra's algorithm works, but not necessarily whether the graphs in this thesis will be directed or undirected, whether you write edges as $uv$, $\{u,v\}$, or $(u,v)$, or which specific formulation of Dijkstra's algorithm you may be referring to.

It is sometimes difficult to decide what to put in the preliminaries, what to leave out completely, and what to introduce later.
As a rule of thumb:
If a concept is only used once, then this concept is best introduced just before it is used.
If a concept is used several times throughout your document, it should be introduced in the preliminaries.
If a concept is very important to your document, then it should probably be introduced in a separate chapter and not in the preliminaries. For example, if you are implementing an algorithm from a paper, then it is best to describe the algorithm and its properties in its own chapter.

  % Add your own chapters here:
  % \include{Sources/mychapter}
  \chapter{Conclusion}

Here, you should briefly reiterate your main results.
Do not simply copy/paste them, but restate them at a high level and reflect on them critically.
Did you achieve what you set out to do?
What limitations does your approach have?
What else could/should be done in a future project?
Which problems are left open by this thesis?


  % This ensures that the subsequent sections are being included as root
  % items in the bookmark structure of your PDF reader.
  \bookmarksetup{startatroot}
  \backmatter

  % The acronyms and glossary are optional.
  % \begingroup
  %   \let\clearpage\relax
  %   \glsaddall
  %   \printglossary[type=\acronymtype]
  %   \newpage
  %   \printglossary
  % \endgroup

  % The index is optional
  % \printindex

  % The bibliography is required.
  \bibliographystyle{alphaurl}% typically plainurl or alphaurl
  \bibliography{Thesis}

\end{document}
