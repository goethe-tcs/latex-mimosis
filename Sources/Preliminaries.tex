\chapter{Preliminaries}

Try to make your thesis self-contained!
You should include here all the preliminary material (notation, definitions, etc.) that is necessary to understand your thesis.
As a rule of thumb, you can assume that your reader is a competent computer scientist who has finished their Master's degree.
For example, the reader knows all standard undergraduate material in theoretical computer science and/or discrete mathematics.
However, you cannot assume that the reader knows the notation you are going to use or has read any specific paper that you may have read.

For example, you can assume that the reader knows what a graph is and how Dijkstra's algorithm works, but not necessarily whether the graphs in this thesis will be directed or undirected, whether you write edges as $uv$, $\{u,v\}$, or $(u,v)$, or which specific formulation of Dijkstra's algorithm you may be referring to.

It is sometimes difficult to decide what to put in the preliminaries, what to leave out completely, and what to introduce later.
As a rule of thumb:
If a concept is only used once, then this concept is best introduced just before it is used.
If a concept is used several times throughout your document, it should be introduced in the preliminaries.
If a concept is very important to your document, then it should probably be introduced in a separate chapter and not in the preliminaries. For example, if you are implementing an algorithm from a paper, then it is best to describe the algorithm and its properties in its own chapter.
